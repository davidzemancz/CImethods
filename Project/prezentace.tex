\documentclass[aspectratio=169]{beamer}
\usetheme{Madrid}
\usecolortheme{default}

\usepackage[utf8]{inputenc}
\usepackage[czech]{babel}
\usepackage{graphicx}
\usepackage{booktabs}
\usepackage{amsmath}
\usepackage{tikz}
\usepackage{pgfplots}
\pgfplotsset{compat=1.18}

\definecolor{cvutblue}{RGB}{0,101,189}
\setbeamercolor{structure}{fg=cvutblue}

\title{Optimalizace rozložení skladu pomocí evolučních algoritmů se surrogate modely}
\subtitle{Feature Selection a Ensemble metody}
\author{David Zeman}
\institute{ČVUT FEL}
\date{2026}

\begin{document}

\begin{frame}
\titlepage
\end{frame}

\begin{frame}{Obsah}
\tableofcontents
\end{frame}

%===============================================================================
\section{Úvod a motivace}
%===============================================================================

\begin{frame}{Problém MAPD (Multi-Agent Pickup and Delivery)}
\begin{columns}
\begin{column}{0.6\textwidth}
\textbf{Problém:}
\begin{itemize}
    \item Sklad s~mřížkovým rozložením (9×9)
    \item 49 pozic pro zboží, 4 agenti
    \item Agenti sbírají objednávky a doručují k~výdejním místům
    \item Distribuce objednávek: \textbf{Zipf} (některé položky jsou populárnější)
\end{itemize}

\vspace{0.5cm}
\textbf{Cíl:}
\begin{itemize}
    \item Najít optimální \textbf{rozložení zboží} ve skladu
    \item Maximalizovat \textbf{průchodnost} (throughput)
    \item Populární zboží blíže k~okrajům $\Rightarrow$ rychlejší obsluha
\end{itemize}
\end{column}
\begin{column}{0.4\textwidth}
\centering
\begin{tikzpicture}[scale=0.5]
    % Grid
    \draw[step=1, gray, thin] (0,0) grid (7,7);
    % Edge (delivery points)
    \fill[blue!20] (0,0) rectangle (7,1);
    \fill[blue!20] (0,6) rectangle (7,7);
    \fill[blue!20] (0,1) rectangle (1,6);
    \fill[blue!20] (6,1) rectangle (7,6);
    % Inner area
    \fill[yellow!30] (1,1) rectangle (6,6);
    % Labels
    \node at (3.5, 0.5) {\scriptsize Výdejní místa};
    \node at (3.5, 3.5) {\scriptsize Zboží};
    % Popular items (red dots)
    \fill[red] (1.5, 1.5) circle (0.2);
    \fill[red] (5.5, 1.5) circle (0.2);
    \fill[orange] (2.5, 2.5) circle (0.15);
\end{tikzpicture}

\vspace{0.3cm}
\scriptsize Červená = populární zboží
\end{column}
\end{columns}
\end{frame}

\begin{frame}{Výzvy optimalizace}
\begin{block}{Problém prohledávacího prostoru}
\begin{itemize}
    \item Layout = permutace 49 položek $\Rightarrow$ \textbf{49! $\approx 6 \times 10^{62}$} možností
    \item Jedna simulace trvá \textbf{2--3 sekundy}
    \item Evoluční algoritmus potřebuje \textbf{stovky až tisíce} evaluací
\end{itemize}
\end{block}

\vspace{0.5cm}
\begin{block}{Řešení: Surrogate model}
\begin{itemize}
    \item Aproximace fitness funkcí pomocí \textbf{strojového učení}
    \item Extrakce \textbf{40 features} z~layoutu (vzdálenosti, distribuce, kongesce, ...)
    \item Predikce throughputu \textbf{bez simulace} $\Rightarrow$ 1000× rychlejší
    \item Hybridní přístup: občasné reálné evaluace pro aktualizaci modelu
\end{itemize}
\end{block}
\end{frame}

%===============================================================================
\section{Feature Selection}
%===============================================================================

\begin{frame}{Extrahované features (40 celkem)}
\begin{columns}
\begin{column}{0.5\textwidth}
\textbf{Distance-based:}
\begin{itemize}
    \item Vážený průměr vzdálenosti k~okraji
    \item Směrodatná odchylka vzdáleností
    \item Max vzdálenost top 10\% položek
    \item Průměr top 5/20 položek
\end{itemize}

\vspace{0.3cm}
\textbf{Position-based:}
\begin{itemize}
    \item Počet populárních položek na okraji
    \item Podíl v~rozích
    \item Kvadrantové rozložení (NW, NE, SW, SE)
\end{itemize}
\end{column}
\begin{column}{0.5\textwidth}
\textbf{Distribution-based:}
\begin{itemize}
    \item Entropie vzdáleností
    \item Gini koeficient
    \item Pravděpodobnostní hmota na okraji
\end{itemize}

\vspace{0.3cm}
\textbf{Congestion/clustering:}
\begin{itemize}
    \item Skóre kongesce
    \item Lokální hustota populárních položek
    \item Bottleneck skóre
    \item Path overlap
\end{itemize}
\end{column}
\end{columns}
\end{frame}

\begin{frame}{Důležitost features (Random Forest)}
\begin{columns}
\begin{column}{0.55\textwidth}
\textbf{Top 10 nejdůležitějších features:}
\begin{enumerate}
    \item \texttt{edge\_prob\_mass} (0.142)
    \item \texttt{dist\_gini} (0.059)
    \item \texttt{quad\_SE} (0.050)
    \item \texttt{spread\_x} (0.050)
    \item \texttt{quad\_NW} (0.044)
    \item \texttt{hotspot} (0.041)
    \item \texttt{dispersion} (0.040)
    \item \texttt{skewness} (0.038)
    \item \texttt{quadrant\_balance} (0.037)
    \item \texttt{quad\_SW} (0.037)
\end{enumerate}
\end{column}
\begin{column}{0.45\textwidth}
\begin{block}{Klíčové zjištění}
\texttt{edge\_prob\_mass} -- pravděpodobnostní hmota na okraji -- je \textbf{nejdůležitější prediktor} throughputu.

\vspace{0.3cm}
Populární položky blízko okraje $\Rightarrow$ vyšší throughput
\end{block}
\end{column}
\end{columns}
\end{frame}

\begin{frame}{Vliv počtu features na kvalitu modelu}
\begin{columns}
\begin{column}{0.5\textwidth}
\centering
\begin{tikzpicture}[scale=0.8]
\begin{axis}[
    xlabel={Počet features (K)},
    ylabel={$R^2$ skóre},
    legend pos=north east,
    grid=major,
    ymin=0, ymax=0.25,
    width=8cm, height=6cm
]
\addplot[blue, mark=o, thick] coordinates {
    (5, 0.192) (10, 0.198) (15, 0.170) (20, 0.167) (25, 0.145) (30, 0.130) (40, 0.072)
};
\addlegendentry{F-regression}

\addplot[red, mark=square, thick] coordinates {
    (5, 0.166) (10, 0.171) (15, 0.151) (20, 0.147) (25, 0.154) (30, 0.116) (40, 0.072)
};
\addlegendentry{Mutual Info}

\addplot[green!60!black, mark=triangle, thick] coordinates {
    (5, 0.187) (10, 0.145) (15, 0.133) (20, 0.120) (25, 0.114) (30, 0.093) (40, 0.072)
};
\addlegendentry{RF Importance}
\end{axis}
\end{tikzpicture}
\end{column}
\begin{column}{0.5\textwidth}
\begin{block}{Výsledky}
\begin{itemize}
    \item Optimální počet features: \textbf{K = 10}
    \item $R^2$ se zvyšuje z~0.072 na \textbf{0.198}
    \item Příliš mnoho features $\Rightarrow$ overfitting
    \item F-regression dosahuje nejlepších výsledků
\end{itemize}
\end{block}

\vspace{0.3cm}
\begin{alertblock}{Zlepšení}
Redukce z~40 na 10 features: \\
\textbf{+175\% zlepšení} $R^2$ skóre
\end{alertblock}
\end{column}
\end{columns}
\end{frame}

%===============================================================================
\section{Ensemble modely}
%===============================================================================

\begin{frame}{Ensemble metody}
\begin{columns}
\begin{column}{0.5\textwidth}
\begin{block}{Voting Ensemble}
\begin{itemize}
    \item Kombinace 3 modelů:
    \begin{itemize}
        \item Ridge Regression
        \item Random Forest
        \item Gradient Boosting
    \end{itemize}
    \item Výstup = \textbf{průměr} predikcí
    \item Robustní, rychlý
\end{itemize}
\end{block}
\end{column}
\begin{column}{0.5\textwidth}
\begin{block}{Stacking Ensemble}
\begin{itemize}
    \item Stejné 3 základní modely
    \item \textbf{Meta-learner} (Ridge) kombinuje predikce
    \item Trénován pomocí cross-validation
    \item Lepší predikce, složitější trénink
\end{itemize}
\end{block}
\end{column}
\end{columns}

\vspace{0.5cm}
\centering
\begin{tikzpicture}[scale=0.7, every node/.style={font=\scriptsize}]
    % Input
    \node[draw, rectangle, minimum width=2cm] (input) at (0,0) {Features (10)};

    % Base models
    \node[draw, rectangle, fill=blue!20] (ridge) at (3,1) {Ridge};
    \node[draw, rectangle, fill=green!20] (rf) at (3,0) {RF};
    \node[draw, rectangle, fill=orange!20] (gbm) at (3,-1) {GBM};

    % Meta-learner
    \node[draw, rectangle, fill=purple!20, minimum width=1.5cm] (meta) at (6,0) {Meta-learner};

    % Output
    \node[draw, rectangle, minimum width=2cm] (output) at (9,0) {Predikce};

    % Arrows
    \draw[->] (input) -- (ridge);
    \draw[->] (input) -- (rf);
    \draw[->] (input) -- (gbm);
    \draw[->] (ridge) -- (meta);
    \draw[->] (rf) -- (meta);
    \draw[->] (gbm) -- (meta);
    \draw[->] (meta) -- (output);
\end{tikzpicture}
\end{frame}

\begin{frame}{Porovnání modelů -- $R^2$ skóre}
\centering
\begin{tabular}{lcc}
\toprule
\textbf{Model} & \textbf{Features} & \textbf{$R^2$} \\
\midrule
\rowcolor{green!20} Voting Ensemble & 10 & $0.166 \pm 0.056$ \\
Baseline GP & 40 & $0.152 \pm 0.040$ \\
Ridge (selected) & 10 & $0.145 \pm 0.048$ \\
Baseline RF & 40 & $0.145 \pm 0.038$ \\
RF (selected) & 10 & $0.136 \pm 0.055$ \\
Stacking Ensemble & 10 & $0.123 \pm 0.019$ \\
GBM (selected) & 10 & $0.084 \pm 0.099$ \\
Baseline Ridge & 40 & $0.072 \pm 0.075$ \\
Baseline Linear & 40 & $0.027 \pm 0.105$ \\
\bottomrule
\end{tabular}

\vspace{0.5cm}
\begin{block}{Poznámka}
Voting Ensemble s~10 features překonává všechny baseline modely se 40 features!
\end{block}
\end{frame}

%===============================================================================
\section{Experimenty a výsledky}
%===============================================================================

\begin{frame}{Nastavení experimentů}
\begin{columns}
\begin{column}{0.5\textwidth}
\textbf{Parametry skladu:}
\begin{itemize}
    \item Velikost: $9 \times 9$
    \item Počet zboží: 49
    \item Počet agentů: 4
    \item Planner: A*
    \item Distribuce: Zipf 1.2
\end{itemize}
\end{column}
\begin{column}{0.5\textwidth}
\textbf{Parametry EA:}
\begin{itemize}
    \item Velikost populace: 20
    \item Počet generací: 50
    \item Reálná evaluace: každých 5 generací
    \item Počáteční vzorky: 300
    \item Simulační kroky: 300
\end{itemize}
\end{column}
\end{columns}

\vspace{0.5cm}
\begin{block}{Hybridní přístup}
\begin{enumerate}
    \item Sběr 300 vzorků pro trénink surrogate modelu
    \item Evoluční algoritmus používá surrogate pro rychlé evaluace
    \item Každých 5 generací: reálná evaluace pro aktualizaci modelu
\end{enumerate}
\end{block}
\end{frame}

\begin{frame}{Výsledky -- Porovnání metod}
\centering
\begin{tabular}{lccc}
\toprule
\textbf{Metoda} & \textbf{Fitness} & \textbf{Real evals} & \textbf{Čas} \\
\midrule
Random (best) & 0.4800 & -- & -- \\
Random (mean) & 0.3399 & -- & -- \\
Greedy & 0.4400 & -- & -- \\
\midrule
EA bez surrogate & 0.5500 & 1020 & 2608s \\
\rowcolor{yellow!30} EA + Voting Ensemble & 0.5600 & 168 & 451s \\
\rowcolor{green!30} EA + Stacking Ensemble & \textbf{0.5800} & 168 & 630s \\
\bottomrule
\end{tabular}

\vspace{0.5cm}
\begin{columns}
\begin{column}{0.5\textwidth}
\begin{block}{Speedup}
\begin{itemize}
    \item Voting: \textbf{5.79×} rychlejší
    \item Stacking: \textbf{4.14×} rychlejší
\end{itemize}
\end{block}
\end{column}
\begin{column}{0.5\textwidth}
\begin{block}{Nejlepší metoda}
\textbf{Stacking Ensemble}
\begin{itemize}
    \item Fitness: 0.5800 (+5.5\%)
    \item 83.5\% redukce evaluací
\end{itemize}
\end{block}
\end{column}
\end{columns}
\end{frame}

\begin{frame}{Konvergence EA}
\centering
\begin{tikzpicture}[scale=0.9]
\begin{axis}[
    xlabel={Generace},
    ylabel={Best Fitness},
    legend pos=south east,
    grid=major,
    width=12cm, height=6.5cm,
    ymin=0.3, ymax=0.6
]
% EA bez surrogate
\addplot[blue, thick] coordinates {
    (0, 0.43) (10, 0.533) (20, 0.533) (30, 0.533) (40, 0.55) (50, 0.55)
};
\addlegendentry{Bez surrogate}

% Voting
\addplot[red, thick] coordinates {
    (0, 0.477) (10, 0.477) (20, 0.56) (30, 0.56) (40, 0.56) (50, 0.56)
};
\addlegendentry{Voting Ensemble}

% Stacking
\addplot[green!60!black, thick] coordinates {
    (0, 0.49) (10, 0.49) (20, 0.497) (30, 0.58) (40, 0.58) (50, 0.58)
};
\addlegendentry{Stacking Ensemble}

% Baselines
\addplot[orange, dashed] coordinates {(0, 0.44) (50, 0.44)};
\addlegendentry{Greedy}

\addplot[gray, dashed] coordinates {(0, 0.48) (50, 0.48)};
\addlegendentry{Random best}
\end{axis}
\end{tikzpicture}
\end{frame}

\begin{frame}{Fitness vs. výpočetní čas}
\centering
\begin{tikzpicture}[scale=0.9]
\begin{axis}[
    xlabel={Čas (s)},
    ylabel={Best Fitness},
    legend pos=south east,
    grid=major,
    width=12cm, height=6.5cm,
    ymin=0.3, ymax=0.6,
    xmax=2800
]
% EA bez surrogate
\addplot[blue, thick] coordinates {
    (0, 0.43) (520, 0.533) (1040, 0.533) (1560, 0.533) (2080, 0.55) (2608, 0.55)
};
\addlegendentry{Bez surrogate}

% Voting (rychlejší)
\addplot[red, thick] coordinates {
    (0, 0.477) (90, 0.477) (180, 0.56) (270, 0.56) (360, 0.56) (451, 0.56)
};
\addlegendentry{Voting Ensemble}

% Stacking
\addplot[green!60!black, thick] coordinates {
    (0, 0.49) (126, 0.49) (252, 0.497) (378, 0.58) (504, 0.58) (630, 0.58)
};
\addlegendentry{Stacking Ensemble}
\end{axis}
\end{tikzpicture}

\vspace{0.3cm}
\textbf{Klíčový závěr:} Surrogate metody dosahují lepších výsledků za zlomek času!
\end{frame}

%===============================================================================
\section{Závěr}
%===============================================================================

\begin{frame}{Shrnutí výsledků}
\begin{columns}
\begin{column}{0.5\textwidth}
\begin{block}{Feature Selection}
\begin{itemize}
    \item Redukce 40 $\rightarrow$ 10 features
    \item Zlepšení $R^2$: +175\%
    \item Klíčová feature: \texttt{edge\_prob\_mass}
\end{itemize}
\end{block}

\begin{block}{Ensemble modely}
\begin{itemize}
    \item Voting: rychlý, robustní
    \item Stacking: nejlepší predikce
    \item Překonávají single modely
\end{itemize}
\end{block}
\end{column}
\begin{column}{0.5\textwidth}
\begin{block}{EA s~surrogate}
\begin{itemize}
    \item Fitness: +5.5\% vs bez surrogate
    \item Speedup: 4--6× rychlejší
    \item 83.5\% méně reálných evaluací
\end{itemize}
\end{block}

\begin{alertblock}{Hlavní přínos}
Kombinace \textbf{feature selection} a \textbf{ensemble surrogate} umožňuje efektivní optimalizaci rozložení skladu s~výraznou úsporou výpočetního času.
\end{alertblock}
\end{column}
\end{columns}
\end{frame}

\begin{frame}{Možná rozšíření}
\begin{itemize}
    \item \textbf{Adaptivní feature selection} -- dynamický výběr features během evoluce
    \item \textbf{Transfer learning} -- přenos znalostí mezi různými velikostmi skladů
    \item \textbf{Multi-objective optimalizace} -- throughput vs. kolize vs. energetická náročnost
    \item \textbf{Online learning} -- průběžná adaptace surrogate modelu
    \item \textbf{Větší sklady} -- škálovatelnost na reálné warehouse rozměry
\end{itemize}

\vspace{0.5cm}
\begin{block}{Praktické aplikace}
\begin{itemize}
    \item E-commerce fulfillment centra (Amazon, Alza, ...)
    \item Automatizované sklady s~AGV roboty
    \item Optimalizace logistických operací
\end{itemize}
\end{block}
\end{frame}

\begin{frame}
\centering
\Huge Děkuji za pozornost!

\vspace{1cm}
\Large Otázky?

\vspace{1cm}
\normalsize
\texttt{david.zeman@fel.cvut.cz}
\end{frame}

\end{document}
